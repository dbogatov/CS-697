% cSpell:ignore blfootnote baidu Wakefield

\begin{frame}{Disclaimers}
	
	\begin{alertblock}{Disclaimer}
		Academic dishonesty in any form is strongly discouraged!
	\end{alertblock}

	\begin{alertblock}{Disclaimer}
		As a member of academic community (and especially as a Teaching Fellow) you are obligated to report any academic misconduct.
	\end{alertblock}

\end{frame}

\section{Introduction}

	\begin{frame}{The plan}
		
		We are going to cover the following

		\begin{itemize}
			\item
				Formal definition
			\item 
				Types of academic dishonesty
				\begin{itemize}
					\item 
						How to avoid
					\item 
						Interesting cases
				\end{itemize}
			\item 
				Prosecution mechanisms
		\end{itemize}

	\end{frame}

	\begin{frame}{Definition}
		
		\begin{block}{Definition}
			\textbf{Academic dishonesty} or academic misconduct is \emph{any} type of cheating that occurs in relation to a formal academic exercise.\blfootnote{
				\cite{bcc} % chktex 2
			}
		\end{block}	

	\end{frame}

	\begin{frame}{Types of misconduct}

		\begin{description}
			\item[Plagiarism]
				The adoption or reproduction of ideas or words or statements of another person without due acknowledgment.
			\item[Fabrication]
				The falsification of data, information, or citations in any formal academic exercise.
			\item[Deception]
				Providing false information to an instructor concerning a formal academic exercise.
			\item[Cheating]
				Any attempt to give or obtain assistance in a formal academic exercise (like an examination) without due acknowledgment.
			\item[Sabotage]
				Acting to prevent others from completing their work.
		\end{description}\blfootnote{
			\cite{bcc} % chktex 2
		}

	\end{frame}

\section{Types of misconduct}

	\subsection{Plagiarism}

		\begin{frame}{Plagiarism}
			
			\begin{block}{Definition}
				Representing the work of another as one’s own.\blfootnote{
					\cite{bu-code} % chktex 2
				}
			\end{block}	

		\end{frame}

		\begin{frame}{Plagiarism}
			
			Includes but is not limited to

			\begin{itemize}
				\item 
					Submitting someone's work as your own
				\item 
					Using quotations, but not citing the source
				\item 
					Copying so many words or ideas from a source that it makes up the majority of your work, whether you give credit or not
				\item 
					Giving incorrect information about the source of a quotation
				\item 
					More elegant: giving URL to not working, password-protected or paid resource
				\item 
					$\ldots$
			\end{itemize}\blfootnote{
				\cite{plagiarism-org-definition},~\cite{wiki-plagiarism} and~\cite{dishonesty-paper} % chktex 2
			}

		\end{frame}

		\begin{frame}{Plagiarism}
			
			Some caveats

			\begin{itemize}
				\item 
					Self-plagiarism
				\item 
					Common knowledge
				\item 
					Copyrighted material
				\item 
					Paraphrasing
				\item 
					Wikipedia
					\begin{itemize}
						\item 
							In most cases, Wikipedia itself is not an original source
					\end{itemize}
				\item 
					$\ldots$
			\end{itemize}

		\end{frame}

		\begin{frame}{Plagiarism}
			
			Preventing plagiarism

			\begin{itemize}
				\item 
					Consult with your instructor
				\item 
					When in doubt, cite sources
				\item 
					Know how to paraphrase
					\begin{itemize}
						\item 
							change both the words and the sentence structure
						\item 
							still requires citing!
					\end{itemize}
				\item 
					Analyze and evaluate your sources
					\begin{itemize}
						\item 
							Make sure you are citing the \emph{original} source
					\end{itemize}
				\item 
					$\ldots$
			\end{itemize}\blfootnote{
				\cite{plagiarism-org-preventing} % chktex 2
			}

		\end{frame}

		\begin{frame}[standout]

			Plagiarism case

			% Professor's story about the student who copied from the roommate's computer
			% before her freshman year
		\end{frame}

	\subsection{Fabrication}

		\begin{frame}{Fabrication}
			
			\begin{block}{Definition}
				\textbf{Misrepresentation or falsification of data} presented for surveys, experiments, reports, etc.\blfootnote{
					\cite{bu-code} % chktex 2
				}

				Includes but is not limited to

				\begin{itemize}
					\item
						Citing authors that do not exist
					\item 
						Citing interviews that never took place
					\item
						Citing field work that was not completed
					\item 
						Manual fitting of experiment result to obtain sound conclusions
						\begin{itemize}
							\item 
								case later in the presentation
						\end{itemize}
				\end{itemize}

			\end{block}

		\end{frame}

		\begin{frame}{Fabrication}
			
			Preventing fabrication

			\begin{itemize}
				\item 
					Consult with your instructor
				\item 
					Clearly explain all aspects of the experiment, survey, etc.
				\item 
					Replicate your experiment and make sure other people can replicate it as well
				\item 
					$\ldots$
			\end{itemize}

		\end{frame}

		\begin{frame}[standout]
			
			Fabrication case \\
			Andrew Jeremy Wakefield

			% https://en.m.wikipedia.org/wiki/Andrew_Wakefield

		\end{frame}

	\subsection{Deception}

		\begin{frame}{Deception}
				
			RQ1: \emph{In what ways do college students report deceiving their instructors?}\blfootnote{
				\cite{deception-survey} % chktex 2
			}

			\begin{description}
				\item[28\%]
					Academic Misconduct
					% Break academic rule of university or instructor
				\item[26\%]
					Late Work
				\item[21\%] 
					Attendance
				\item[8\%]
					False Reaction
				\item[*\%]
					Others Cheating / Nonverbal Feedback / No Deception
			\end{description}

		\end{frame}

		\begin{frame}{Deception}
				
			RQ2: \emph{What are students’ motives for deception in the classroom?}\blfootnote{
				\cite{deception-survey} % chktex 2
			}

			\begin{description}
				\item[82\%]
					Better Grade / Avoid Grade Deduction
				\item[8\%]
					Impression Management
				\item[*\%] 
					Avoid Involvement / Altruism
			\end{description}

		\end{frame}

		\begin{frame}{Deception}
				
			RQ3: \emph{What methods do college students rely on when deceiving instructors?}\blfootnote{
				\cite{deception-survey} % chktex 2
			}

			\begin{description}
				\item[50\%]
					Falsification
				\item[31\%]
					Concealment
				\item[8\%] 
					Half Truth / Half Concealment
				\item[*\%]
					Denial / Incorrect Inference Dodge / Misdirecting
			\end{description}

		\end{frame}

		\begin{frame}{Deception}
				
			RQ4: \emph{How often do students report obtaining their goals through deception?}\blfootnote{
				\cite{deception-survey} % chktex 2
			}

			\begin{description}
				\item[92\%]
					Goals Accomplished
				\item[*\%]
					Did not help
			\end{description}

		\end{frame}

	\subsection{Cheating and Sabotage}

		\begin{frame}{Cheating}
				
			Subset of Stanford's Academic Cheating Fact Sheet

			\begin{itemize}
				\item 
					\textbf{Cheating is a major problem}
				\item 
					Grades, rather than education, have become the major focus of many students.
				\item 
					Many students feel that their individual honesty in academic endeavors will not affect anyone else.
				\item 
					Students cheat because they see others cheat and they think they will be unfairly disadvantaged.
				\item 
					In most cases cheaters don't get caught. If caught, they seldom are punished severely, if at all.
				\item
					Top influences: others doing it and faculty member doesn't seem to care.
			\end{itemize}\blfootnote{
				\cite{cheating-fact-sheet} % chktex 2
			}

		\end{frame}

		\begin{frame}{Sabotage}
				
			Northern Illinois University's examples

			\begin{itemize}
				\item 
					Destroying another person's work
				\item 
					Not contributing to a collaborative effort adequately
				\item 
					Withholding information when it should be shared with others
				\item 
					Revealing confidential data about another person's project
				\item 
					Falsely accuse others of academic dishonesty
				\item
					Destroying books and materials in the library, laboratory, etc
			\end{itemize}\blfootnote{
				\cite{sabotage-examples} % chktex 2
			}

		\end{frame}

		\begin{frame}[standout]
			
			Cheating case \\
			Baidu Team Is Barred From A.I. Competition~\cite{baidu-cheating}

			% https://en.m.wikipedia.org/wiki/Andrew_Wakefield

		\end{frame}

\section{Prosecution}

	\begin{frame}{Prosecution}

		From BU Academic Conduct Code

		\begin{itemize}
			\item 
				Faculty must report
			\item 
				First-Time Offender vs Repeating Offender
			\item 
				Approved Admission of Academic Misconduct Form
				\begin{itemize}
					\item 
						Grade penalty
				\end{itemize}
			\item
				Academic Conduct Committee
				\begin{itemize}
					\item 
						No penalty
					\item 
						Reprimand
					\item 
						Disciplinary probation
					\item
						Suspension
					\item
						Expulsion
				\end{itemize}
		\end{itemize}\blfootnote{
			\cite{bu-code} % chktex 2
		}

	\end{frame}
