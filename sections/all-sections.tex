% cSpell:ignore DBLP

\begin{frame}{Disclaimers}
	
	\begin{alertblock}{Disclaimer}
		Academic dishonesty in any form is strongly discouraged!
	\end{alertblock}

	\begin{alertblock}{Disclaimer}
		As a member of academic community (and especially as Teaching Fellow) your obligated to report any academic misconduct.
	\end{alertblock}

\end{frame}

\section{Intro}

	\begin{frame}{The plan}
		
		We are going to cover the following

		\begin{itemize}
			\item
				Formal definition
			\item 
				Types of academic dishonesty
				\begin{itemize}
					\item 
						How to avoid
				\end{itemize}
			\item 
				Prosecution mechanisms
			\item
				Interesting cases
		\end{itemize}

	\end{frame}

	\begin{frame}{Definition}
		
		\begin{block}{Definition}
			\textbf{Academic dishonesty} or academic misconduct is \emph{any} type of cheating that occurs in relation to a formal academic exercise.\blfootnote{
				Cited from~\cite{bcc}.
			}
		\end{block}	

	\end{frame}

	\begin{frame}{Types of misconduct}

		\begin{description}
			\item[Plagiarism]
				The adoption or reproduction of ideas or words or statements of another person without due acknowledgment.
			\item[Fabrication]
				The falsification of data, information, or citations in any formal academic exercise.
			\item[Deception]
				Providing false information to an instructor concerning a formal academic exercise.
			\item[Cheating]
				Any attempt to give or obtain assistance in a formal academic exercise (like an examination) without due acknowledgment.
			\item[Sabotage]
				Acting to prevent others from completing their work.
		\end{description}\blfootnote{
			Cited from~\cite{bcc}.
		}

	\end{frame}

\section{Plagiarism}

	\begin{frame}{Plagiarism}
		
		\begin{block}{Definition}
			Representing the work of another as one’s own.\blfootnote{
				Cited from~\cite{bu-code}.
			}
		\end{block}	

	\end{frame}

	\begin{frame}{Plagiarism}
		
		Includes but is not limited to

		\begin{itemize}
			\item 
				Submitting someone's work as your own
			\item 
				Using quotations, but not citing the source
			\item 
				Copying so many words or ideas from a source that it makes up the majority of your work, whether you give credit or not
			\item 
				Giving incorrect information about the source of a quotation
			\item 
				More elegant: giving URL to not working, password-protected or paid resource
			\item 
				$\ldots$
		\end{itemize}\blfootnote{
			Cited from~\cite{plagiarism-org-definition},~\cite{wiki-plagiarism} and~\cite{dishonesty-paper}.
		}

	\end{frame}

	\begin{frame}{Plagiarism}
		
		Some caveats

		\begin{itemize}
			\item 
				Self-plagiarism
			\item 
				Common knowledge
			\item 
				Copyrighted material
			\item 
				Paraphrasing
			\item 
				Wikipedia
				\begin{itemize}
					\item 
						In most cases, Wikipedia itself is not an original source
				\end{itemize}
			\item 
				$\ldots$
		\end{itemize}

	\end{frame}

	\begin{frame}{Plagiarism}
		
		Preventing plagiarism

		\begin{itemize}
			\item 
				Consult with your instructor
			\item 
				When in doubt, cite sources
			\item 
				Know how to paraphrase
				\begin{itemize}
					\item 
						change both the words and the sentence structure
					\item 
						still requires citing!
				\end{itemize}
			\item 
				Analyze and evaluate your sources
				\begin{itemize}
					\item 
						Make sure you are citing the \emph{original} source
				\end{itemize}
			\item 
				$\ldots$
		\end{itemize}\blfootnote{
			Cited from~\cite{plagiarism-org-preventing}.
		}

	\end{frame}

\section{Fabrication}

	\begin{frame}{Fabrication}
		
		\begin{block}{Definition}
			\textbf{Misrepresentation or falsification of data} presented for surveys, experiments, reports, etc.\blfootnote{
				Cited from~\cite{bu-code}.
			}

			Includes but is not limited to

			\begin{itemize}
				\item
					Citing authors that do not exist
				\item 
					Citing interviews that never took place
				\item
					Citing field work that was not completed
				\item 
					Manual fitting of experiment result to obtain sound conclusions
					\begin{itemize}
						\item 
							case later in the presentation
					\end{itemize}
			\end{itemize}

		\end{block}

	\end{frame}

	\begin{frame}{Fabrication}
		
		Preventing fabrication

		\begin{itemize}
			\item 
				Consult with your instructor
			\item 
				Clearly explain all aspects of the experiment, survey, etc.
			\item 
				Replicate your experiment and make sure other people can replicate it as well
			\item 
				$\ldots$
		\end{itemize}

	\end{frame}


