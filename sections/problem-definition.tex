% cSpell:ignore Maas EECS

\section{Problem definition}

	\begin{frame}{Notations}
		
		\begin{itemize}
			\item 
				The client fetches/stores data in atomic units --- \emph{blocks} --- of size $B$ bytes each
			\item 
				Let $N$ be the working set --- number of distinct data blocks stored in ORAM
		\end{itemize}

		\note{
			Let us start with some notation.
			Assume that all data comes in atomic units called blocks and assume that we have $N$ distinct blocks in ORAM\@.
		}
	\end{frame}

	\begin{frame}{Simplicity}
		
		\begin{quote}
			We aim to provide an extremely simple ORAM construction in contrast with previous work
		\end{quote}
		
		Can then be implemented in hardware~\cite{Maas:EECS-2014-89}.

		\note{
			On of the goals is simplicity.
			It not only allows us to analyze system easily.
			In case of PathORAM, it is so simple that it has been implemented in pure hardware.

			One of the works that puts PathORAM on the silicon is written by \textcite{Maas:EECS-2014-89}. % chktex 8
		}
	\end{frame}

	\begin{frame}{Security definitions}
		
		\begin{block}{Formal security definition}
			
			Data request sequence of length $M$, where each $\text{op}_i$ denotes a $\text{read}(\text{a}_i)$ or a $\text{write}(\text{a}_i, \text{data}_I)$ operation
			
			\[
				\vec{y} := ( (\text{op}_M, \text{a}_M, \text{data}_M), \ldots , (\text{op}_1, \text{a}_1, \text{data}_1) )
			\]

			Let $A(\vec{y})$ denote the (possibly randomized) sequence of accesses to the remote storage given the sequence of data requests $\vec{y}$.

		\end{block}

		\note{
			Let us define what we mean by access pattern.
			We mean a sequence of operations --- writes and reads --- on some blocks with identifiers $\text{a}_i$ reading or writing some $\text{data}_i$.
			Say we have $M$ operations in sequence.
		}
	\end{frame}

	\begin{frame}{Security definitions}
		
		\begin{block}{Formal security definition}
			
			An ORAM construction is said to be secure if

			\begin{itemize}
				\item 
					for any two data request sequences $\vec{y}$ and $\vec{z}$ of the same length, their access patterns $A(\vec{y})$ and $A(\vec{z})$ are computationally indistinguishable by anyone but the client
				\item 
					the ORAM construction is correct with probability $p \ge 1 - negl(|\vec{y}|)$.
			\end{itemize}

		\end{block}

		\note{
			The most important one.
			We call a construction secure if an adversary cannot tell, which sequence of operations really went through the construction.
			Indistinguishability is a strong requirement --- it says that we leak absolutely no information about our patterns.
			And indeed, we need to correctly respond to client requests.
			Since we allow randomized algorithms, we set another requirement on error probability.
			It has to be negligibly small.
		}
	\end{frame}
